\documentclass[12pt, a4paper, sans]{moderncv}

\linespread{1.1}

\usepackage{url}

% colors: black, blue, green, grey, orange & purple
% styles: banking, casual, classic & oldstyle
\moderncvtheme[blue]{banking}

% character encoding
\usepackage[utf8]{inputenc}
\usepackage[english]{babel}
\usepackage{datetime}

\newcommand{\revision}{\twodigit\day\twodigit\month\the\year}
\renewcommand*{\namefont}{\fontsize{34}{38}\mdseries\upshape}

\usepackage{fancyhdr}
\pagestyle{fancy}
\fancyfoot[RE, RO]{\footnotesize{rev. \revision}}

% adjust the page margins
\usepackage[scale=.85]{geometry}
\setlength{\hintscolumnwidth}{3cm}
\AtBeginDocument{\recomputelengths}

% personal data
\firstname{Welington}
\familyname{Veiga}

\phone[mobile]{\texttt{+55 32 98847-7526}}
%\homepage{http://agonzalezro.github.io}
\email{welington.veiga@gmail.com}

\social[github]{welingtonveiga}
\social[twitter]{welingtonveiga}
\social[linkedin]{welingtonveiga}

\begin{document}
\makecvtitle


%
% PROFESSIONAL EXPERIENCE
%

\subsection{Experiência}

\cventry{Agosto/2017 -- Atual}
{Professor}
{Centro de Ensino Superior de Juiz de Fora}{Juiz de Fora (MG)}{}
{ 
-Professor dos cursos de Sistemas de Informação e Engenharia de Software nas disciplinas Laboratório de Web Sites Dinâmicos, Gerenciamento de Configurações e Versionamento e Projetos.
}


\cventry{Março/2017 -- Atual}
{Professor}
{PUC Minas}{Juiz de Fora (MG)}{}
{ 
-Professor da disciplina de Gerenciamento de Riscos do Curso de Pós-Graduação em Gerência de Projetos.
}

\cventry{Outubro/2016 -- Outubro/2017}
{Gerente de Desenvolvimento}
{Affero Lab}{Juiz de Fora (MG)}{}
{
- Gerenciamentos de área de desenvolvimento composta por times ágeis de desenvolvimento, incluindo coordenador, supervisores, especialistas, analistas de desenvolvimento,  qualidade e estagiários, incentivando um ambiente dinâmico, criativo, aberto e ao mesmo tempo altamente produtivo, com entregas frequentes, com foco em qualidade e excelência técnica.\newline{}
- Participação em projetos com uso de plataformas como J2EE, stack Spring, NodeJS, com bancos de dados relacionais (MySQL, SQL Server, PostgreSQL) e não relacionais (AWS DynamoDB, Redis), serviços de fila (RabbitMQ), integrações (Apache Camel, Spring Integration), mecanismos de cache (Memcached, EhCache, Infinispan), ferramentas para deploy automatizado, integração contínua, testes automatizados (unitários e de integração).\newline{}
- Coach técnico de gestores e colaboradores com sugestões de literatura, conteúdos na web e acompanhamento de sua evolução.}

\cventry{Agosto/2016 -- Atual}
{Professor}
{Faculdade Metodista Granbery}{Juiz de Fora (MG)}{}
{
- Professor do curso de Sistemas de Informação, lecionando disciplinas como Desenvolvimento para Dispositivos Móveis, Desenvolvimento Orientado à Objetos, Desenvolvimento de Aplicações Web e Inteligência Artificial.}


\cventry{Fevereiro/2014 -- Setembro/2016}
{Coordenador de Equipe de TI}
{Affero Lab}{Juiz de Fora (MG)}{}
{
- Liderança técnica e coordenação de times ágeis de desenvolvimento, trabalhando para manutenção de um ambiente dinâmico, criativo, aberto e ao mesmo tempo altamente produtivo, com entregas frequentes, foco em qualidade e excelência técnica.\newline{}
- Participação em projetos com uso de plataformas como J2EE, stack Spring, NodeJS, com bancos de dados relacionais (MySQL, SQL Server, PostgreSQL) e não relacionais (AWS DynamoDB, Redis), serviços de fila (RabbitMQ), integrações (Apache Camel), mecanismos de cache (Memcached, EhCache, Infinispan), ferramentas para deploy automatizado, integração contínua, testes automatizados (unitários e de integração).\newline{}
- Uso de soluções de computação em nuvem para infraestrutura das aplicações, com foco em escalabilidade horizontal e alta disponibilidade.\newline{}
- Coach técnico dos colaboradores com sugestões de literatura, conteúdos na web e acompanhamento de sua evolução.\newline{}}

\cventry{Fevereiro/2013 -- Janeiro/2014}
{Analista de Sistemas Sênior}
{Affero Lab}{Juiz de Fora (MG)}{}
{- Migração de sistema legado para ambiente clusterizado usando infraestrutura da Amazon (AWS).\\
- Desenvolvimento ágil utilizando o framework Scrum, atuando como Scrum Master.\\
- Especificação e desenvolvimento de integrações entre sistemas de diferentes fornecedores.\\
- Interação constante com equipes de atendimento, comercial, projetos e cliente.}

\cventry{Abril/2012 -- Janeiro/2013}
{Analista de Sistemas Pleno}
{Affero Lab}{Juiz de Fora (MG)}{}
{- Evolução e manutenção de aplicações J2EE utilizando Struts e Hibernate em software de avaliação e desenvolvimento de recursos humanos em cenários de alta carga, aplicando metodologias ágeis, TDD com JUnit, Mockito, PowerMockito, Selenium e testes de carga.}

\cventry{Junho/2011 -- Maio/2012}
{Analista de Sistemas}
{Projetus TI}{Juiz de Fora (MG)}{}
{- Manutenção e desenvolvimento de sistemas em Java (Maven, Spring, JasperReports, Hibernate, EclipseLink/JPA, Tomcat, PostgreSQL, Adobe Flex ) na área Contábil, atuando na adequação do sistema às constantes atualizações na legislação.}

\cventry{Maio/2009 -- Fevereiro/2011}
{Programador Júnior}
{Aprimorar Desenvolvimento}{Juiz de Fora (MG)}{}
{- Desenvolvimento de um sistema de comunicação voltado para produtividade em PHP utilizando o Framework Symfony e o ORM Doctrine sobre o SGBD MySql. Desenvolvimento de interfaces web ricas com uso intenso de JavaScript. Participação na análise de requisitos, especificação e modelagem dos projetos.\newline{}
}


%
% EDUCATION
%
\subsection{Formação acadêmica}
\cventry{2018--2022 (Previsto)}
{Doutorando em Informátiva - Programa de Pós Graduação em Informática}{Universidade Federal do Estado do Rio de Janeiro -- UNIRIO}{}{}{}


\cventry{2014--2016}
{Mestrado em Ciência da Computação - Engenharia de Software}{Universidade Federal de Juiz de Fora -- UFJF}{}{}{}

\cventry{2008--2012}
{Bacharelado em Ciência da Computação}{Universidade Federal de Juiz de Fora -- UFJF}{}{}{}

\subsection{Certificações}

\cventry{2013--2017}
{Certified ScrumMaster}{Scrum Alliance, Inc}{}{}{}



%
% LANGUAGES
%

\subsection{Idiomas}
\cvdoubleitem{\textbf{Inglês}}{Intermediário.}{\textbf{Português}}{Nativo.}

%
% MISC
%
%\section{Other interesting information}
%\cvlistitem{Co-organizer of the \url{golang.co.uk} conference. The first Go
%conference in UK. Also started PyGrunn monthly at the Netherlands and currently
%co-organizing the Go London User Group meetups.}

%\cvlistitem{Speaker in several biz and tech talks: J2ME talk at
%XGN, to I+D+i managers at \hyphenation{ADEuropa}ADEuropa,
%about FLOSS with CENATIC, Django at DJUGL, GTK at Paylogic, Thrift at PyGrunn\ldots}

\end{document}
